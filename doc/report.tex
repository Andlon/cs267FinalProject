\documentclass[11pt, oneside]{scrreprt}   	% use "amsart" instead of "article" for AMSLaTeX format
\usepackage{geometry}                		% See geometry.pdf to learn the layout options. There are lots.
\geometry{letterpaper}                   		% ... or a4paper or a5paper or ... 
%\geometry{landscape}                		% Activate for rotated page geometry
\usepackage[parfill]{parskip}    		% Activate to begin paragraphs with an empty line rather than an indent
\usepackage{graphicx}				% Use pdf, png, jpg, or eps§ with pdflatex; use eps in DVI mode
								% TeX will automatically convert eps --> pdf in pdflatex		
\usepackage{amssymb}

%SetFonts

%SetFonts


\title{CS267 Final Project Report}
\subtitle{Parallelizing Empirical Variograms for Large Data Sets}
\author{Andreas Borgen Longva\\
  \texttt{longva@berkeley.edu}
  \and
  Heather Savoy\\
  \texttt{frystacka@berkeley.edu}}
\date{May 7, 2015}							% Activate to display a given date or no date

\begin{document}
\maketitle

\section{Project Description}
%What is a variogram, its application, why not yet parallel, comparison to particle simulation
%Note on how this will actually be used in research
%How we tested for correctness

\section{Implementations}
%Overview of different implementations and their goals 
%Any assumptions we made for all?
  \subsection{Serial}
  %Brief description
  %pseudo-code
  %Expected performance
  \subsection{Naive MPI}
  %Brief description
  %pseudo-code
  %Expected performance
  \subsection{Communication Optimal}
  %Brief description
  %pseudo-code
  %Expected performance
  \subsection{Computation Optimal}
 %Brief description
  %pseudo-code
  %Expected performance
\section{Results}
  %Plot of n versus time for all implementations
  %Extrapolation of how long my data set would take with each?
  %Actual result for my data set

\section{Conclusion}
%Which implementation(s) worked best? Always the case?
%What can be learned from the results to my dataset
%How this can be made more general (for future applications)
%Can now fit models and make simulations (for which algorithms already exist)


\end{document}  